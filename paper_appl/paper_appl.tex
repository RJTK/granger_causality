\documentclass[12pt]{article}

% TODO: Are we, in this setting, immune to confounders?  We are removing
%       all bi-directional edges, so it seems likely that we are?  Can we state
%       any theorem about this?

%% bibliography stuff -- this needs come before the preamble inclusion
\usepackage[backend=bibtex,sorting=none]{biblatex}
\usepackage{enumitem}
\usepackage{etex,etoolbox}
\usepackage{hyperref}
\usepackage{fullpage}
\setlength{\marginparwidth}{2cm}
\usepackage{todonotes}
\usepackage{booktabs}
\usepackage{multirow}
\bibliography{\string~/Documents/academics/global_academics/global_bib.bib}
% \usepackage{apxproof}

% \bibliography{\string~global_bib.bib}

\def\gc{\overset{\text{GC}}{\rightarrow}}  % Granger causality arrow
\def\ngc{\overset{\text{GC}}{\nrightarrow}}  % Negated Granger causality arrow
\def\pwgc{\overset{\text{PW}}{\rightarrow}}  % Pairwise Granger causality arrow
\def\npwgc{\overset{\text{PW}}{\nrightarrow}}  % Negated pairwise Granger causality arrow
\def\te{\overset{\mathcal{T}}{\rightarrow}}  % Transfer entropy arrow
\def\gcg{\mathcal{G}}  % Granger-causality graph
\def\gcge{\mathcal{E}}  % Graph edges
\def\VAR{\mathsf{VAR}}  % VAR(p) model
\def\B{\mathsf{B}}  % Filter B
\def\wtB{\widetilde{\B}}  % General filter B
\def\A{\mathsf{A}}  % Filter A
\def\H{\mathcal{H}}  % Hilbert space

\newcommand{\linE}[2]{\hat{\E}[#1\ |\ #2]}  % Linear projection
\newcommand{\linEerr}[2]{\xi[#1\ |\ #2]}  % Error of linear projection
\newcommand{\pa}[1]{pa(#1)}  % Parents of a node
\newcommand{\anc}[1]{\mathcal{A}(#1)}  % Ancestors of a node
\newcommand{\ancn}[2]{\mathcal{A}_{#1}(#2)}  % nth ancestors of a node
\newcommand{\gpn}[2]{gp_{#1}(#2)}  % nth generation grandparents
\newcommand{\wtalpha}[2]{\widetilde{\alpha}(#1, #2)}  % Some notation for lem:pwgc_anc
\newcommand{\dist}[2]{\mathsf{d}(#1, #2)}  % Distance between things
\newcommand{\gcgpath}[2]{#1 \rightarrow \cdots \rightarrow #2}  % A shorter path command

% \usepackage{fullpage}
\usepackage{framed}

% Figures
\usepackage{graphicx}
\usepackage{caption}
% \usepackage{subcaption}
\usepackage{wrapfig}
\usepackage{svg}

% Math packages, theorem definitions and numbering
\usepackage{amsmath}
\usepackage{amssymb}
\usepackage{amsthm}
\usepackage{mathrsfs} % Fancy scripted font
\usepackage{bm}  % Bold math
\usepackage{centernot}  % \centernot\implies looks better

% Misc packages
\usepackage[linesnumbered, ruled, vlined]{algorithm2e}
% \usepackage{algorithm2e} %{algorithm} environment
\usepackage{soul}  % \hl highlighting
\usepackage{color}
\usepackage{mathtools}  % For my \ceil function

% Theorems (with italics)
\theoremstyle{plain}  % Style definition removes italics
\newtheorem{theorem}{Theorem}
\newtheorem{corollary}{Corollary}
\newtheorem{proposition}{Proposition}
\newtheorem{lemma}{Lemma}

\theoremstyle{definition}
\newtheorem{remark}{Remark}
\newtheorem{definition}{Definition}
\newtheorem{example}{Example}
\newtheorem{assumption}{Assumption}

% keywords
\providecommand{\keywords}[1]{\textbf{\textit{Keywords---}} #1}

% General
\def\defeq{\overset{\Delta}{=}}  % Equal with triangle
\def\cl{\mathsf{cl\ }}  % Closure
\newcommand{\sgn}[1]{\mathsf{sgn}(#1)}  % sign function

% Calculus
\def\d{\mathsf{d}}  % Differential operator

% Functions
\def\ln{\mathsf{ln\ }}  % Natural logarithm
\DeclarePairedDelimiter{\ceil}{\lceil}{\rceil}  % Ceiling

% Probability
\def\H{\mathcal{H}}  % Hilbert space
\def\E{\mathbb{E}}  % Expectation
\def\Var{\text{Var}}  % Variance
\def\P{\mathbb{P}}  % Probability Measure
\def\F{\mathcal{F}}  % A sigma algebra
\def\sX{\mathcal{X}}  % Another sigma algebra
\def\KL{\mathbf{D}_{KL}}  % KL divergence
\def\bF{\mathbf{F}}  % Whole F-meas space
\def\GP{\mathcal{GP}}  % Gaussian process

% Standard sets
\def\Z{\mathbb{Z}}  % Set of integers
\def\R{\mathbb{R}}  % Set of real numbers
\def\C{\mathbb{C}}  % Set of complex numbers
\def\N{\mathbb{N}}  % Set of natural numbers
\def\ball{\mathbb{B}}  % Open ball
\def\clball{\overline{\ball}}  % Closed ball

% Linear algebra
\def\rk{\mathsf{rk }}  % The rank
\def\tr{\mathsf{tr }}  % The trace
\def\T{\mathsf{T}}  % Transpose notation
\def\c{\mathsf{c}}  % complement
\def\dg{\mathsf{dg }}   %  Diagonal vector of a matrix
\def\Dg{\mathsf{Dg }}   %  Diagonal matrix from a vector
\def\ind{\mathbf{1}}  % Ones vector or indicator
\def\matvec{\textbf{vec}}  % Vector operator
\def\<{\langle}  % < Inner product
\def\>{\rangle}  % > Inner product
\newcommand{\inner}[2]{\langle #1, #2 \rangle}  % Inner product
\newcommand{\innerT}[2]{#1^\T #2}  % Inner product for finite vectors

% Convex analysis
\def\conv{\mathsf{conv }}  % Convex hull
\def\prox{\mathsf{prox }}  % Proximity operator

% -----------------
% The given symbol or text (\text{mytext}) in a circle
% To be used always in math mode
\newcommand{\circlesign}[1]{ 
    \mathbin{
        \mathchoice
        {\buildcirclesign{\displaystyle}{#1}}
        {\buildcirclesign{\textstyle}{#1}}
        {\buildcirclesign{\scriptstyle}{#1}}
        {\buildcirclesign{\scriptscriptstyle}{#1}}
    } 
}

\newcommand\buildcirclesign[2]{%
    \begin{tikzpicture}[baseline=(X.base), inner sep=0, outer sep=0]
    \node[draw,circle] (X)  {\ensuremath{#1 #2}};
    \end{tikzpicture}%
}
% -----------------

% \input{\string~latex_preamble}

\graphicspath{{../figures/}}

\title{A Study in Pairwise Testing Heuristics for Granger-causal Network Estimation}
\author{R. J. Kinnear \\
  \small\href{mailto:ryan@kinnear.ca}{ryan@kinnear.ca} \\
  \small\url{https://github.com/RJTK} \and R. R. Mazumdar \\
  \small\href{mailto:mazum@uwaterloo.ca}{mazum@uwaterloo.ca}}

\begin{document}
\maketitle
\abstract{Following recent theoretical developments of
  Granger-causality graph estimation for ``strongly causal'' causality
  graphs, we develop in detail a computationally and statistically
  efficient heuristic for Granger-causality estimation which is based
  purely on pairwise causality testing.  We perform detailed
  comparisons against state of the art algorithms in the LASSO family
  through a simulation study, and demonstrate the capacity for scaling
  to thousands of nodes in a time series network.  We develop an
  application example for EEG data where we accurately discriminate
  between subjects based purely on the Granger-causality graphs
  inferred from data by our algorithms, demonstrating that meaningful
  features are being captured in the Granger-causalilt graph topology,
  as well as providing evidence that individuals exhibit meaningfully
  distinct EEG causality structures, which may be a result of
  independent scientific interest.}

\keywords{causality graph, EEG, Granger-causality, LASSO, network
  learning, time series, vector autoregression}

\paragraph{Acknowledgement}
We acknowledge the support of the Natural Sciences and Engineering Research Council of Canada (NSERC), [funding reference number 518418-2018].  Cette recherche a été financée par le Conseil de recherches en sciences naturelles et en génie du Canada (CRSNG), [numéro de référence 518418-2018].

\clearpage

% \tableofcontents
% \clearpage

\section{Introduction and Review}
\label{sec:introduction}
Following recent theoretical developments \cite{my_GC_paper}
establishing that, under certain topological assumptions on the
underlying causality graph, it is possible to accurately recover the
true causality graph by way of pairwise causality tests alone.  That
is, there are algorithms which are immune to the effects of
confounders when applied to certain causality graph structures.  In
this paper we provide simulation evidence that, given structural
assumptions, such algorithms can dramatically outperform state of the
art LASSO-type estimators.  Moreover, even in cases where structural
assumptions do not in fact hold, using similar algorithms as pure
heuristics can still exhibit superior performance.  In addition to the
potential for gains in statistical power, that our algorithms make use
only of pairwise causality testing improves the scalability (both in
terms of the asymptotics, as well as through trivial parallelization)
of Granger-causality testing up to thousands of nodes.

Granger-causality is a notion is leveraged in a variety of
applications e.g. in Neuroscience as a means of recovering
interactions amongst brain regions \cite{bressler2011wiener},
\cite{anna_paper2008}, \cite{david2008identifying}; in the study of
the dependence and connectedness of financial institutions
\cite{NBERw16223}; gene expression networks \cite{Fujita2007},
\cite{methods_for_inferring_gene_regulatory_networks_from_time_series_expression_data},
\cite{grouped_graphical_granger_modelling_for_gene_expression_regulatory_networks_discovery},
\cite{discovering_graphical_Granger_causality_using_the_truncating_lasso_penalty};
and power system design \cite{Misyrlis2016450}, \cite{yuan2014root}.

Granger-causality can generally be formulated by searching for the
``best'' graph structure consistent with observed data, which is in
general an extremely challenging problem (i.e. it may be framed as a
best subset selection problem, see \cite{bss_mio},
\cite{hastie_bss_comp}), moreover, the comparison of quality between
different structures, and hence the notion of ``best'' needs
qualification.  In applications where we are interested merely in
minimizing the mean squared error of a linear one-step-ahead
predictor, then we will be satisfied with an entirely dense graph of
connections, since each edge can only serve to reduce estimation
error.  However, since the number of edges scales quadratically in $n$
(the number of nodes) it becomes imperative to infer a sparse
causality graph for large systems, both to avoid overfitting observed
data, as well as to aid the interpretability of the results.

A fairly early approach to the problem in the context of large systems
is provided by \cite{bach2004learning}, where the authors apply a
local search heuristic to the Whittle likelihood with an AIC
penalization.  The local search heuristic where at each iteration an
edge is either added, removed, or reversed is a common approach to
combinatorial optimization due to it's simplicity, but is liable to
get stuck in shallow local minima.

A second and wildly successful heuristic is the LASSO regularizer
\cite{tibshirani1996regression}, which can be understood as a natural
convex relaxation to penalizing the count of the non-zero edges.  The
LASSO enjoys fairly strong theoretical guarantees
\cite{wainwright2009sharp}, extending largely to the case of
stationary time series data with a sufficiently fast rate of
dependence decay \cite{basu2015} \cite{wong2016lasso}
\cite{autoregressive_process_modelling_via_the_lasso_procedure}, and
variations on the LASSO have been applied in a number of different
time series contexts as well as Granger-causality
\cite{DBLP:journals/corr/HallacPBL17} \cite{haufe2008sparse}
\cite{bolstad2011causal} \cite{he2013stationary}
\cite{grouped_graphical_granger_modelling_for_gene_expression_regulatory_networks_discovery}.
One of the key improvements to the original LASSO algorithm is the
adaptive (i.e. weighted) ``adaLASSO'' \cite{adaptive_lasso_zou2006},
for which oracle results (i.e. asymptotic support recovery) are
established under less restrictive conditions than for the vanilla
LASSO.  Our experimental comparisons in Section
\ref{sec:empirical_evaluation} are against the
adaLASSO. % Success of the LASSO has lead to
% the famous ``bet on sparsity'' principle ``Use a procedure that does
% well in sparse problems, since no procedure does well in dense
% problems.''  \cite{tibshirani2015statistical}.

% \subsection{Contributions}
% In the context of time series data, sparsity assumptions remain
% important, but there is significant additional structure that may
% arise as a result of considering the topology of the underlying
% Granger-causality graph, which to our knowledge remains largely
% unexplored.  The focus of this paper is to shed light on some of these
% topological questions, in particular, we study a particularly simple
% notion of causality graph topology which we term ``strongly causal''
% and show that stationary times series whose underlying causality graph
% has this structure satisfy natural intuitive notions of ``information
% flow'' through the graph.  Moreover, we show that such graphs are
% perfectly recoverable with only \textit{pairwise} Granger-causality
% tests, which would otherwise suffer from serious confounding problems.
% Our finite sample results are based on simulations, where we show
% promising results for these graph topologies where our algorithm
% performs substantially better in our simulation setup than do
% competing LASSO algorithms, even for graphs that do not exactly
% satisfy our strongly-causal topology assumptions.

% In the case of gene expression networks, we show examples from the
% literature which suggest our concept of a ``strongly causal graph''
% topology may have application in this field (see Section
% \ref{sec:strongly_causal_graphs}).

% The principle contributions of this paper are as follows: firstly, in
% section \ref{sec:theory} we study \textit{pairwise} Granger-causality
% relations, providing novel theorems connecting the structure of the
% causality graph to the pairwise ``causality flow'' in the system, as
% well as an interpretation in terms of the graph topology of the
% sparsity pattern of matrices arising in the Wold decomposition,
% generalizing in some sense the notion of ``feedback-free'' processes
% studied by \cite{caines1975feedback} in close connection with
% Granger-causality.  We establish sufficient conditions (sections
% \ref{sec:strongly_causal_graphs}, \ref{sec:persistent_systems}) under
% which a fully conditional Granger-causality graph can be recovered
% from pairwise tests alone (sec \ref{sec:pairwise_algorithm}).
% Secondly, we propose in section \ref{sec:structure_learning} a graph
% search heuristic which implements our theoretical results to finite
% data samples, specifying and summarizing appropriate methods for
% hypothesis testing (section \ref{sec:pairwise_hypothesis_testing}),
% model order selection (section \ref{sec:model_order_selection}),
% computationally efficient estimation (section
% \ref{sec:efficient_model_estimation}), and error rate controls
% (section \ref{sec:error_rate_control}).  Our heuristics are compared
% against the adaLASSO algorithm in Section
% \ref{sec:empirical_evaluation}.  We stress the scalability of our
% algorithm which is capable of comfortably handling hundreds or
% thousands of nodes on a single machine, as opposed to standard LASSO
% algorithms which do not take advantage of the special structure
% associated with stationary time series data.  In section
% \ref{sec:application} we develop an example application where a
% classifier is constructed to accurately discriminate between subjects
% in an EEG study based only on the Granger-causality graph topology
% inferred by our algorithms.  Concluding remarks on further open
% problems and extentions are provided in Section \ref{sec:conclusion}.

\section{Granger Causality}
\label{sec:theory}
For the purposes of this paper, we provide an expediant and simplified
definition of Granger-causality.  For a more detailed exposition see
e.g. \cite{my_GC_paper}, \cite{granger1969investigating},
\cite{geweke1982measurement}, \cite{lutkepohl2005new}.  Suppose that $x(t)$ is an
$n-$dimensional wide-sense stationary time series generated by the
following (stable) $VAR(\infty)$ model

\begin{equation}
  \label{eq:var_model}
  x(t) = \sum_{\tau = 1}^\infty B(\tau) x(t - \tau) + v(t).
\end{equation}

\begin{definition}[Granger-causality]
  \label{def:granger_causality}
  We will say that the $j$th component of $x$, namely $x_j(t)$,
  Granger-causes component $x_i(t)$ (with respect to $x$) and write
  $j \gc i$ if $\exists \tau \in \Z_{++}$ such that $B_{ij}(\tau) \ne 0$.
\end{definition}

This definition captures the notion that $x_j$ provides information
about $x_i$ which is not available through any of the other components
of $x$.  The motivation being that, since an effect must follow it's
cause, that $x_j$ ``causes'' $x_i$.  This intuition can also be given
interpretation as the ``flow'' of ``information'' or ``energy''.

\begin{definition}[Granger-causality Graph]
  We define the Granger-causality graph $\gcg = ([n], \gcge)$ to be
  the directed graph formed on $n$ vertices where an edge
  $(j, i) \in \gcge$ if and only if $x_j$ Granger-causes $x_i$ (with
  respect to $x$).  That is,
  $$(j, i) \in \gcge \iff j \in \pa{i} \iff x_j \gc x_i.$$
\end{definition}

Referring back to the definition (Def. \ref{def:granger_causality}) of
Granger-causality it is important that the concept be understood with
respect to a particular universe of observations.  If $x_j \gc x_i$
with respect to $x_{-k}$, it may not hold with respect to $x$.  For
example, $x_k$ may be a common ancestor which when observed,
completely explains the connection from $x_j$ to $x_i$.  We introduce
the pairwise definition of Granger-causality as a special case:

\begin{definition}[Pairwise Granger-causality]
  We will say that $x_j$ pairwise Granger-causes $x_i$ and write
  $x_j \pwgc x_i$ if $x_j$ Granger-causes $x_i$ with respect only to
  $(x_i, x_j)$.
\end{definition}

This notion is of interest for a variety of reasons.  From a purely
conceptual standpoint, we will see how the notion can in some sense
capture the idea of ``flow of information'' in the underlying graph,
in the sense that if $j \in \anc{i}$ we expect that $j \pwgc i$.  It
may also be useful for reasoning about the conditions under which
\textit{unobserved} components of $x(t)$ may or may not interfere with
inference in the actually observed components.  Finally, motivated
from a practical standpoint to analyze causation in large systems, we
will seek to construct practical estimation procedures based purely on
pairwise causality tests since the computation of such pairwise
relations is somewhat easier.

\subsection{Strongly Causal Graphs and Persistent Systems}
\label{sec:strongly_causal_graphs}
In this section we define the notion of a strongly causal graph, and
the implications for pairwise causality testing.

\begin{definition}[Strongly Causal \cite{my_GC_paper}]
  \label{def:strongly_causal}
  We will say that a Granger-causality graph $\gcg$ is
  \textit{strongly causal} if it is acyclic and there is at most 1
  directed path between any two nodes.  Strongly Causal Graphs will be
  referred to as SCGs.
\end{definition}

  Examples of strongly causal graphs include directed trees (or
  forests), DAGs where each node has at most one parent, and complete
  bipartite graphs.  Examples of graphs arising in practice that appear
  to be ``close'' to strongly causal (i.e. few edges violate the condition)
  appear to be available in biology, in particular, the authors of
  \cite{discovering_graphical_Granger_causality_using_the_truncating_lasso_penalty}
  cite an example of the so called ``transcription regulatory network
  of \textit{E.coli}'', and
  \cite{learning_genome_scale_regulatory_networks} study a much larger
  regulatory network of \textit{Saccharomyces cerevisiae}.  These
  networks, which we reproduce\footnote{Figure \ref{fig:gene_network1}
    is reproduced under the Creative Commons Attribution
    Non-Commercial License
    (\url{http://creativecommons.org/licenses/by-nc/2.5}) and Figure
    \ref{fig:gene_network2} under the Creative Commons Attribution
    License (\url{https://creativecommons.org/licenses/by/4.0/})} in
  figure \ref{fig:gene_networks} appear to have at most a small number
  of edges which violate the strong-causality condition.

  \begin{figure}[h]
    \centering
    \caption{Transcription Regulatory Networks}
    \label{fig:gene_networks}
    \begin{subfigure}[b]{0.45\textwidth}
      \caption{\textit{E.Coli} Network of
        \cite{discovering_graphical_Granger_causality_using_the_truncating_lasso_penalty}}
      \label{fig:gene_network1}
      \includegraphics[width=\linewidth, height=\linewidth]{ecoli_regulatory_network.png}
    \end{subfigure}
    \begin{subfigure}[b]{0.45\textwidth}
      \caption{\textit{Saccharomyces cerevisiae} Network of
        \cite{learning_genome_scale_regulatory_networks}}
      \label{fig:gene_network2}
      \includegraphics[width=\linewidth, height=\linewidth]{huge_gene_network.png}
    \end{subfigure}
  \end{figure}

  In order to exclude the possibility of certain types of cancellation
  in the $\VAR$ system of Equation \eqref{eq:var_model}, we require
  the system to have a property we have deemed ``persistence'' in
  \cite{my_GC_paper}, the intuition being that each node in the system
  maintains some ``memory''.
  
  \begin{definition}[Persistent]
    \hl{TODO:}I can provide the diagonalizability condition for
    $\VAR(p)$ models, but what about when $p = \infty$?
    \todo{persistence for $\VAR(\infty)$}
  \end{definition}

  The key property of persistent and strongly causal systems is that
  they are in some sense immune to confounders.  That is, if $i \pwgc j$
  \textit{and} $j \pwgc i$, then $i \ngc j$ and $j \ngc i$.  That is,
  bidirectional pairwise causation (which would otherwise be the
  result of hidden confounders) is not possible.

  \begin{theorem}[Pairwise Recovery \cite{my_GC_paper}]
    \label{thm:scg_recovery}
    If the Granger-causality graph $\gcg$ for persistent process
    $x(t)$ is a strongly causal DAG, then $\gcg$ can be inferred from
    pairwise causality tests, given an oracle for pairwise causation.

    \todo{Contain statement on convergence of the heuristic?}
    % \todo{Make this precisise} Moreover, given $T < \infty$ samples from
    % $x(t)$ then Algorithm \ref{alg:finite_pwgc} will correctly return
    % $\gcg$ with probability $1$ as $T \rightarrow \infty$.
  \end{theorem}

\section{Finite Sample Graph Recovery}
\label{sec:structure_learning}
In \cite{my_GC_paper} we prove Theorem \ref{thm:scg_recovery} by
providing an algorithm which, given an oracle for pairwise
Granger-causality, will correctly recover $\gcg$.  In this section,
we develop in detail appropriate statistical methods for
implementing this algorithm on a finite data sample.  When applied
in practice, the algorithm is a heuristic\footnote{Recall that
  Granger-causality is necessarily based on heuristics when $n$ is
  modestly large, as the complexity of searching over causality
  graphs scales as $O(2^{n^2})$}, since we do not know apriori
whether the causality graph of $x(t)$ is strongly causal (or if
$x(t)$ is persistent).  However, we will see in Section
\ref{sec:empirical_evaluation} that Algorithm \ref{alg:finite_pwgc}
outperforms LASSO-based heuristics, even when $\gcg$ does not
exactly satisfy the strong causality assumption.

In the appendix Section \ref{sec:pairwise_hypothesis_testing} we describe
straightforward methods for implementing Algorithm
\ref{alg:finite_pwgc}.  In particular, we consider it important to be
able to test pairwise causation in a computationally efficient manner,
since $O(n^2)$ pairwise tests are needed.  In practice, carrying out
each pairwise test consumes the vast majority of computational time,
though since this step is trivially parallelizable, we are able to
scale the estimation procedure to over $1000$ nodes on a personal
computer.

\begin{algorithm}
    \SetKwInOut{Input}{input}
    \SetKwInOut{Output}{output}
    \SetKwInOut{Initialize}{initialize}
    \DontPrintSemicolon

    \BlankLine
    \caption{Finite Sample Pairwise Graph Recovery (PWGC)}
    \label{alg:finite_pwgc}

    \Input{Estimates of pairwise Granger-causality statistics $F_{ij}$
      (eqn \ref{eqn:gc_statistics}).  Matrix of edge probabilities $P_{ij}$ (eqn \ref{eqn:edge_inclusion_probability}).  Hypothesis testing threshold $\delta$ chosen via the Benjamini-Hochberg criterion (Section \ref{sec:error_rate_control})}
    \Output{A strongly causal graph $\widehat{\gcg}$}
    \Initialize{$S = [n]$  \texttt{\# unprocessed nodes}\\
      $E = \emptyset$  \texttt{\# edges of }$\widehat{\gcg}$\\
      $k = 1$ \texttt{\# a counter used only for notation}}

    \BlankLine

    $W_\delta \leftarrow \{(i, j)\ |\ P_{ji} > 1 - \delta, F_{ji} > F_{ij}\}$  \texttt{\# candidate edges}\\
    $\mathcal{I}_0 \leftarrow \big(\sum_{j\in S: (j, i) \in W_\delta} P_{ij}, \mathsf{\ for\ }i \in S \big)$ \texttt{\# total node incident probability}\\
    $P_0 \leftarrow \{i \in S\ |\ \mathcal{I}_0(i) < \ceil{\text{min}(\mathcal{I}_0)}\}$ \texttt{\# Nodes with fewest incident edges}\\
    \If{$P_0 = \emptyset$}{
      $P_0 \leftarrow \{i \in S\ |\ \mathcal{I}_0(i) \le \ceil{\text{min}(\mathcal{I}_0)}\}$ \texttt{\# Ensure non-empty}
    }
    \BlankLine

    \While{$S \ne \emptyset$}{
      $S \leftarrow S \setminus P_{k - 1}$ \texttt{\# remove processed nodes}\\
      % $\mathcal{I}_k \leftarrow \big(\sum_{j \in S: (j, i) \in W_\delta} F_{ij}, \mathsf{\ for\ }i \in S \big)$ \texttt{\# intra-}$S$ \texttt{incident strength}\\
      $\mathcal{I}_k \leftarrow \big(\sum_{j\in S: (j, i) \in W_\delta} P_{ij}, \mathsf{\ for\ }i \in S \big)$\\
      $P_k \leftarrow \{i \in S\ |\ \mathcal{I}_k(i) < \ceil{\text{min}(\mathcal{I}_k)}\}$\\
      \If{$P_k = \emptyset$}{
        $P_k \leftarrow \{i \in S\ |\ \mathcal{I}_k(i) \le \ceil{\text{min}(\mathcal{I}_k)}\}$
      }
      \;
      \texttt{\# add strongest edges, maintaining strong causality}\\
      $U_k \leftarrow \bigcup_{r = 1}^k P_{k - r}$ \texttt{\# Include all forward edges}\\
      \For{$(i, j) \in \mathsf{sort}\Big(\{(i, j) \in U_k \times P_k\ |\ (i, j) \in W_\delta\} \mathsf{\ by\ descending\ } F_{ji}\Big)$} {
        \If{$\mathsf{is\_strongly\_causal}(E \cup \{(i, j)\})$} {
          \texttt{\# }$\mathsf{is\_strongly\_causal}$ \texttt{can be implemented by keeping}\\
          \texttt{\# track of ancestor / descendant relationships}\\
          $E \leftarrow E \cup \{(i, j)\}$
        }
      }
      $k \leftarrow k + 1$\\
    }
    \Return{$([n], E)$}
\end{algorithm}

\section{Empirical Evaluation}
\label{sec:empirical_evaluation}
We have implemented our empirical experiments in Python \cite{scipy},
in particular we leverage the LASSO implementation from
\texttt{sklearn} \cite{sklearn} and the random graph generators from
\texttt{networkx} \cite{networkx}.  We run experiments using two
separate graph topologies having $n$ nodes: a strongly causal graph
(SCG) and a directed acyclic graph (DAG).  These are generated
respectively by drawing a random tree and a random Erdos Renyi graph
(with edge probability $q = \frac{2}{n}$ resulting in approximately
the same number of edges for the SCG as for the DAG), then creating a
directed graph by directing edges from lower numbered nodes to higher
numbered nodes.

We populate each of the edges (including self loops) with random
linear filters constructed by placing $5$ transfer function poles
(i.e. $p = 5$) uniformly at random in a disc of radius $3 / 4$ (which
guarantees stability for acyclic graphs).  The resulting system is
driven by i.i.d. Gaussian random noise, each component having random
variance $\sigma_i^2 = 1/2 + r_i$ where $r_i \sim \text{exp}(1/2)$.
We set $p_{\text{max}} = 15$.  Results and representative graphs are
collected in Figures \ref{fig:random_graph_topologies},
\ref{fig:simulation_results_comparison1},
\ref{fig:simulation_results_comparison2},
\ref{fig:simulation_results_scaling_and_small_T},
\ref{fig:simulation_results_dense}.

We compare our results against the adaptive LASSO
\cite{adaptive_lasso_zou2006}, which outperformed substantially both
the LASSO and the grouped LASSO.  Motivated by scaling, we split the
squared error term into separate terms, one for each group of incident
edges on a node, and estimate the collection of $n$ incident filters
$\big\{\B_{ij}(z)\big\}_{j = 1}^n$ that minimizes
$\xi_i^{\text{LASSO}}$ in the following:

\begin{equation}
  \begin{aligned}
  \xi_i^{\text{LASSO}}(\lambda) &= \underset{B}{\text{min}}\ \frac{1}{T}\sum_{t = p + 1}^T\big(x_i(t) - \sum_{\tau = 1}^p\sum_{j = 1}^n B_{i, j}(\tau) x(t - \tau)\big)^2 + \lambda \sum_{\tau = 1}^p \sum_{j = 1}^n |B_{ij}(\tau)|\\
  \xi_i^{\text{LASSO}} &= \underset{\lambda \ge 0}{\text{min}}\ \xi_i^{\text{LASSO}}(\lambda) + \mathsf{BIC}\big(B_i^{\text{LASSO}}(\lambda)\big)\\
  \end{aligned}
\end{equation}

where we are choosing $\lambda$, the regularization parameters, via the BIC.

\begin{figure}
  \centering
  \caption{Representative Random Graph Topologies}
  \label{fig:random_graph_topologies}
  \begin{subfigure}[b]{0.3\textwidth}
    \caption{Random SCG}
    \includegraphics[width=\linewidth]{example_scg.pdf}
  \end{subfigure}
  \begin{subfigure}[b]{0.3\textwidth}
    \caption{Random DAG $(q = \frac{2}{n})$}
    \includegraphics[width=\linewidth]{example_dag.pdf}
  \end{subfigure}
  \begin{subfigure}[b]{0.3\textwidth}
    \caption{Random DAG $(q = \frac{4}{n})$}
    \includegraphics[width=\linewidth]{example_dag_q.pdf}
  \end{subfigure}
\end{figure}

\begin{remark}[Graph Topologies]
  We depict in Figure \ref{fig:random_graph_topologies} the topologies
  of random graphs used in our empirical evaluation.  For values of
  $q$ close to $\frac{2}{n}$, the resulting random graphs tend to have
  a topology which is, at least qualitatively, close to the SCG.  As
  the value of $q$ increases, the random graphs deviate farther from
  the SCG topology, and we therefore expect the LASSO to outperform
  PWGC for larger values of $q$.  This can be observed in Figure
  \ref{fig:simulation_results_dense}, at least where performance is
  measured by the support recovery (i.e. via the MCC).
\end{remark}

\begin{remark}[MCC as a Support Recovery Measurement]
  We apply Matthew's Correlation Coefficient (MCC)
  \cite{matthews1975comparison} as a statistic for measuring support
  recovery performance.  This statistic synthesizes the confusion
  matrix into a single score appropriate for unbalanced labels and is
  calibrated to fall into the range $[-1, 1]$ with $1$ being perfect
  performance, $0$ being the performance of random guessing, and $-1$
  being perfectly opposed.
\end{remark}

\begin{remark}[Error Measurement]
  We estimate the 1-step ahead prediction error by forming the variance matrix estimate

  \begin{equation*}
    \widehat{\Sigma}_v \defeq \frac{1}{T_{\text{out}}} \sum_{t = 1}^{T_{\text{out}}} (x(t) - \widehat{x}(t))(x(t) - \widehat{x}(t))^\T
  \end{equation*}

  on a long stream of out-of-sample data.  We then report the quantity

  \begin{equation*}
    \frac{\ln \tr \widehat{\Sigma}_v}{\ln \tr \Sigma_v}
  \end{equation*}

  where $\widehat{\Sigma}_v = \Sigma_v$ is the best possible performance.
\end{remark}

\begin{figure}
  \centering
  \caption{PWGC Compared Against AdaLASSO \cite{adaptive_lasso_zou2006} (SCG)}
  \label{fig:simulation_results_comparison1}
  \includegraphics[width=0.95\textwidth]{new_lasso_comparison_scg_pmax15_simulation.pdf}

  {\scriptsize Comparison of PWGC and LASSO for $\VAR(p)$ model
    estimation.  We make comparisons against both the MCC and the
    relative log mean-squared prediction error
    $\frac{\ln\tr \widehat{\Sigma}_v}{\ln\tr \Sigma_v}$.  Results
    in Figure \ref{fig:simulation_results_comparison1} are for systems
    guaranteed to satisfy the assumptions required for Theorem
    \ref{thm:scg_recovery}.}
\end{figure}

\begin{figure}
  \caption{PWGC vs adaLASSO (DAG, $q = \frac{2}{n}$)}
  \label{fig:simulation_results_comparison2}
  \includegraphics[width=0.95\textwidth]{new_lasso_comparison_dag_pmax15_simulation.pdf}

  {\scriptsize Figure \ref{fig:simulation_results_comparison2}
    provides results for systems which do not guarantee the
    assumptions of Theorem \ref{thm:scg_recovery}, though the graph
    has a similar level of sparsity.}
\end{figure}

\begin{figure}
  \centering
  \caption{PWGC Scaling and Small Sample Performance}
  \label{fig:simulation_results_scaling_and_small_T}
  \begin{subfigure}[b]{0.45\textwidth}
    \caption{Fixed $T$, increasing $n$ (SCG)}
    \label{fig:simulation_results_scaling}
    \includegraphics[width=\linewidth]{new_increasing_n_simulation.pdf}
  \end{subfigure}
  \begin{subfigure}[b]{0.45\textwidth}
    \caption{MCC Comparison for $T \le 100$}
    \label{fig:small_T_comparison}
    \includegraphics[width=\linewidth]{new_mcc_comparison001.pdf}
  \end{subfigure}

  {\scriptsize Figure \ref{fig:simulation_results_scaling} measures
    support recovery performance as the number of nodes $n$ increases,
    and the edge proportion as well as the number of samples $T$ is
    held fixed.  Remarkably, the degradation as $n$ increases is
    limited, it is primarily the graph topology (SCG or non-SCG) as
    well as the level of sparsity (measured by $q$) which are the
    determining factors for support recovery performance.

    Figure \ref{fig:small_T_comparison} provides a support recovery
    comparison for very small values of, $T$ typical for many
    applications.}
\end{figure}

\begin{figure}
  \caption{Fixed $T, n$, increasing edges $q$ (DAG)}
  \label{fig:simulation_results_dense}
  \includegraphics[width=\linewidth]{dag_increasing_q_small_T_alasso_simulation.pdf}

  {\scriptsize Figure \ref{fig:simulation_results_dense} provides a
    comparison between PWGC and AdaLASSO as the density of graph edges
    (as measured by $q$) increases.  For reference,
    $\frac{2}{n} = 0.04$ has approximately the same level of sparsity
    as the SCGs we simulated.  As $q$ increases, the AdaLASSO
    outperforms PWGC as measured by the MCC.  However, PWGC maintains
    superior performance for 1-step-ahead prediction.  We speculate
    that this is a result of fitting the sparsity pattern recovered by
    PWGC via OLS which directly seeks to optimize this metric, whereas
    the LASSO is encumbered by the sparsity inducing penalty.}
\end{figure}

In reference to figure \ref{fig:simulation_results_comparison1} it
should not be overly surprising that our PWGC algorithm performs
better than the LASSO for the case of a strongly causal graph, since
in this case the assumptions which guarantee the correctness of
Algorithm \ref{alg:finite_pwgc} hold.  However, the performance is
still markedly superior in the case of a more general DAG.  We would
conjecture that a DAG having a similar degree of sparsity as an SCG is
likely to be ``close'' to an SCG.  Figure
\ref{fig:simulation_results_dense} illustrates the severe (expected)
degradation in performance as the number of edges increases while
the number of data samples $T$ remains fixed.  For larger values $q$
in this plot, the number of edges in the graph is comparable to the
number of data samples.

We have also paid close attention to the performance of PWGC in the
very small sample ($T \le 100$) regime (see Figure
\ref{fig:small_T_comparison}), as this is the regime many applications
must contend with.

In regards scalability, we have observed that performing the $O(n^2)$
pairwise Granger-causality calculations consumes the vast majority
($> 90\%$) of the computation time.  Since this step is trivially
parallelizable, our algorithm also scales well with multiple cores or
multiple machines.  Figure \ref{fig:simulation_results_scaling} is a
demonstration of this scalability, where we are able to estimate
graphs having over $1500$ nodes (over $2.25 \times 10 ^6$ possible edges)
using only $T = 500$ data points, granted, an SCG on this many nodes
is extremely sparse.

\begin{table}
  \centering
  \caption{Simulation Results: PWGC vs AdaLASSO}
  \label{taab:simulation_table}

  \begin{tabular}{|ll||ll|ll|ll|}
    \toprule
    &{Algorithm}&alasso&pwgc&alasso&pwgc&alasso&pwgc\\
    &\textbf{Metric}&Err&Err&FDR&FDR&MCC&MCC\\
    \textbf{T}&\textbf{q}&&&&&&\\
    \midrule
    \multirow{4}{*}{\textbf{50}}
    &\textbf{SCG}&1.83&\textbf{1.57}&0.48&\textbf{0.07}&0.47&\textbf{0.57}\\
    &\textbf{0.06}&1.91&\textbf{1.69}&0.54&\textbf{0.09}&0.43&\textbf{0.53}\\
    &\textbf{0.11}&2.89&\textbf{2.58}&0.46&\textbf{0.22}&0.38&0.40\\
    &\textbf{0.46}&8.49&\textbf{7.72}&0.42&0.42&\textbf{0.16}&0.11\\
    \midrule
    \multirow{4}{*}{\textbf{250}}
    &\textbf{SCG}&1.32&\textbf{1.22}&0.27&\textbf{0.07}&0.71&\textbf{0.80}\\
    &\textbf{0.06}&1.44&\textbf{1.28}&0.32&\textbf{0.09}&0.66&\textbf{0.76}\\
    &\textbf{0.11}&2.57&\textbf{2.17}&0.32&\textbf{0.16}&0.53&\textbf{0.57}\\
    &\textbf{0.46}&8.09&\textbf{7.11}&0.39&0.40&\textbf{0.20}&0.14\\
    \midrule
    \multirow{4}{*}{\textbf{1250}}
    &\textbf{SCG}&1.18&\textbf{1.09}&0.40&\textbf{0.06}&0.69&\textbf{0.89}\\
    &\textbf{0.06}&1.24&1.17&0.43&\textbf{0.05}&0.65&\textbf{0.86}\\
    &\textbf{0.11}&2.02&2.10&0.33&\textbf{0.14}&0.62&0.63\\
    &\textbf{0.46}&7.44&\textbf{7.04}&0.40&\textbf{0.30}&\textbf{0.22}&0.19\\
    \bottomrule
  \end{tabular}

  % \begin{tabular}{|ll||ll|ll|ll|}
  %   \toprule
  %   &\textbf{Algorithm}&alasso&pwgc&alasso&pwgc&alasso&pwgc\\
  %   &\textbf{Metric}&Err&Err&FDR&FDR&MCC&MCC\\
  %   \textbf{T}&\textbf{q}&&&&&&\\
  %   \midrule
  %   \multirow{4}{*}{\textbf{50}}
  %   &\textbf{SCG}&1.86&\textbf{1.51}&0.49&\textbf{0.16}&0.47&\textbf{0.63}\\
  %   &\textbf{0.06}&1.92&\textbf{1.67}&0.52&\textbf{0.17}&0.44&\textbf{0.60}\\
  %   &\textbf{0.11}&3.06&\textbf{2.81}&0.47&\textbf{0.34}&0.37&\textbf{0.41}\\
  %   &\textbf{0.46}&8.90&\textbf{7.74}&\textbf{0.42}&0.52&\textbf{0.16}&0.10\\
  %   \cline{1-8}
  %   \multirow{4}{*}{\textbf{250}}
  %   &\textbf{SCG}&1.35&\textbf{1.20}&0.28&\textbf{0.21}&0.70&\textbf{0.77}\\
  %   &\textbf{0.06}&1.40&\textbf{1.27}&0.33&\textbf{0.22}&0.66&\textbf{0.73}\\
  %   &\textbf{0.11}&2.48&\textbf{2.18}&0.31&0.30&\textbf{0.55}&0.52\\
  %   &\textbf{0.46}&7.96&\textbf{6.89}&0.38&0.38&\textbf{0.20}&0.15\\
  %   \cline{1-8}
  %   \multirow{4}{*}{\textbf{1250}}
  %   &\textbf{SCG}&1.15&1.12&0.42&\textbf{0.24}&0.67&\textbf{0.79}\\
  %   &\textbf{0.06}&1.28&1.24&0.45&\textbf{0.24}&0.64&\textbf{0.75}\\
  %   &\textbf{0.11}&2.11&2.05&0.34&\textbf{0.29}&\textbf{0.60}&0.56\\
  %   &\textbf{0.46}&7.12&\textbf{6.79}&0.39&\textbf{0.32}&\textbf{0.24}&0.18\\
  %   \bottomrule
  % \end{tabular}

  % \begin{tabular}{|ll||ll|ll|ll|}
  %   \toprule
  %   &\textbf{Algorithm}&alasso&pwgc&alasso&pwgc&alasso&pwgc\\
  %   &\textbf{metric}&RLMSE&&FDR&&MCC&\\
  %   \textbf{T}&\textbf{q}&&&&&&\\
  %   \midrule
  %   \multirow{4}{*}{\textbf{50}}
  %   &\textbf{SCG}&2.38&\textbf{1.56}&0.49&\textbf{0.16}&0.48&\textbf{0.63}\\
  %   &\textbf{0.06}&2.66&\textbf{1.68}&0.52&\textbf{0.19}&0.45&\textbf{0.60}\\
  %   &\textbf{0.11}&3.91&\textbf{2.68}&0.46&\textbf{0.32}&0.38&\textbf{0.42}\\
  %   &\textbf{0.46}&10.24&\textbf{7.94}&\textbf{0.41}&0.49&\textbf{0.17}&0.11\\
  %   \cline{1-8}
  %   \multirow{4}{*}{\textbf{250}}
  %   &\textbf{SCG}&2.14&\textbf{1.21}&0.30&\textbf{0.20}&0.69&\textbf{0.78}\\
  %   &\textbf{0.06}&2.26&\textbf{1.29}&0.32&\textbf{0.24}&0.66&\textbf{0.73}\\
  %   &\textbf{0.11}&3.60&\textbf{2.12}&0.31&0.29&0.55&0.53\\
  %   &\textbf{0.46}&10.17&\textbf{7.27}&0.40&0.41&\textbf{0.19}&0.14\\
  %   \cline{1-8}
  %   \multirow{4}{*}{\textbf{1250}}
  %   &\textbf{SCG}&1.88&\textbf{1.12}&0.39&\textbf{0.22}&0.70&\textbf{0.81}\\
  %   &\textbf{0.06}&1.88&\textbf{1.21}&0.41&\textbf{0.25}&0.66&\textbf{0.75}\\
  %   &\textbf{0.11}&3.24&\textbf{2.17}&0.34&\textbf{0.28}&\textbf{0.59}&0.56\\
  %   &\textbf{0.46}&9.74&\textbf{6.93}&0.40&\textbf{0.35}&\textbf{0.22}&0.17\\
  %   \bottomrule
  % \end{tabular}

  {\scriptsize Results of Monte Carlo simulations comparing PWGC and
    AdaLASSO for small samples and when the SCG assumption doesn't
    hold.  The superior result is bolded when the difference is
    statistically significant, as measured by
    \texttt{scipy.stats.ttest\_rel}.  100 iterations are run for each
    set of parameters.}
\end{table}

\section{Application}
\label{sec:application}
In this section we apply our methods to a real set of EEG data
obtained from the ``EEG Database Data
Set''\footnote{http://archive.ics.uci.edu/ml/datasets/EEG+Database}
\cite{zhang1995event} on the UCI machine learning repository
\cite{uci_mlr}.  This dataset contains 1 second long measurements of
(64 channel) EEG signals from patients who are given visual stimuli.
The subjects in the study are labeled as being either ``control'' or
``alcoholic'', however, we will ignore this label and instead focus on
\textit{distinguishing between subjects} based on the
Granger-causality graphs inferred from the subject's trials.  Our
reasoning is to focus on the underlying question: ``Does our PWGC
algorithm uncover meaningful Granger-causal connections from EEG
data?''.  Focusing only on the subject label allows us to answer this
question without simultaneously grappeling with the physiological
question of whether alcoholic subjects as a group have discernable
differences in their EEG readings\footnote{Some exploratory analysis
  actually suggests that alcoholic Granger-causality graphs are not
  substantively different}, which is a stronger requirement than
simply that there are meaningful distinctions between the EEG readings
of subjects generally.

The dataset consists of
$\mathcal{D} = \{\big(x^{(i)}(t)\big)_{t = 1}^T, y^{(i)} \}_{i = 1}^N$
where $T = 256$, $x^{(i)}(t) \in \R^{64}$,
$y^{(i)} \in [N_{subjects}]$ (with $N_{subjects} = 119$), and
$N = 10723$.  There are on average 90 trials for each subject,
ranging between 30 and 119.

We have constructed a simple pipeline for discriminating between
subjects by first applying an iterative PWGC algorithm (see Algorithm
\ref{alg:iterated_pwgc}) directly to the EEG data $x^{(i)}$ to obtain
a Granger-causality graph $\widehat{G}^{(i)}$.  We then feed the
vectorized adjacency matrix of $\widehat{G}^{(i)}$ through a
polynomial Kernel multinomial logistic regression model with
parameters fit by cross validation.

\begin{remark}
  The iterated application of PWGC is an example of a heuristic by
  which graph estimates that aren't constrained to being strongly
  causal can be obtained.  We defer to future work the theoretical
  analysis of more sophisticated heuristics, or algorithms appropriate
  for different topological assumptions.
\end{remark}

\begin{algorithm}
  \SetKwInOut{Input}{input}
  \SetKwInOut{Output}{output}
  \SetKwInOut{Initialize}{initialize}
  \DontPrintSemicolon

  \BlankLine
  \caption{Iterated PWGC Heuristic}
  \label{alg:iterated_pwgc}
  
  \Input{Data $x(t)$, maximum number of iterations $N$,
    threshold $r \in (0, 1)$.}
  \Output{Granger-causality graph estimate $\widehat{G}$ obtaind from
    iterated application of PWGC Algorithm \ref{alg:finite_pwgc}}
  \Initialize{$i = 1$, $\widehat{x}_0(t) \leftarrow x(t)$,
    $\sigma_0 \leftarrow Var\ x(t)$}

  \While{$i \le N$} {
    Estimate $\VAR$ model via PWGC on $x_{i - 1}(t)$ to obtain $\widehat{x}_{i - 1}(t)$ and $\widehat{G}_{i - 1}$\\
    $\epsilon_i(t) \leftarrow x_{i - 1}(t) - \widehat{x}_{i - 1}(t)$ \texttt{\# Compute residuals}\\
    $\sigma_i^2 \leftarrow \text{MSE}\big(\epsilon_i(t)\big)$ \texttt{\# Compute MSE}\\
    \If{$\sigma_i^2 > r \sigma_{i - 1}^2$}{
      \texttt{break  \# Insignificant Improvement}
    }
    $i \leftarrow i + 1$
  }
  $\widehat{G} \leftarrow \bigcup_{i} \widehat{G}_{i}$ \texttt{\# Combine PWGC graph estimates}\\
  \Return{$\widehat{G}$}
\end{algorithm}

While it is almost certainly possible to achieve much greater
classification accuracy on this dataset by constructing a classifier
to act directly on $x^{(i)}(t)$, our purpose is to demonstrate that
relevant information is being captured by the graph estimates
$\widehat{G}^{(i)}$.  That is, that the latent structure
$\widehat{G}^{(i)}$ may provide scientifically relevant insights, as
opposed to simply being a feature for classification tasks.

Figures \ref{fig:ar_eeg_example} and \ref{fig:eeg_oos_error}
illustrate the appropriateness of modelling $x(t)$ with $\VAR$ models,
as opposed simply to a collection of $n$ unidimensional
autoregressiive models.  The conclusion being that the
Granger-causality graph estimates are not purely spurious.

\begin{figure}
  \centering
  \begin{subfigure}[b]{0.45\textwidth}
    \caption{$\mathsf{AR}(p)$ Model Example}
    \label{fig:ar_eeg_example}
    \includegraphics[width=\linewidth]{eeg_estimate_example.pdf}
  \end{subfigure}
  \begin{subfigure}[b]{0.45\textwidth}
    \caption{OOS Error for a Single Subject}
    \label{fig:eeg_oos_error}
    \includegraphics[width=\linewidth]{eeg_oos_demonstration.pdf}
  \end{subfigure}

  {\scriptsize Figure \ref{fig:ar_eeg_example} provides an example of
    an autoregressive model fir via the Levinson-Durbin algorithm
    (choosing the system order via the BIC) on a single EEG channel.
    Linear autoregressive models appear qualitatively to be adequate
    for this data.  Figure \ref{fig:eeg_oos_error} demonstrates the
    improvement of a unified $\VAR$ model over independent
    $\mathsf{AR}$ models, providing evidence that intra-node edges
    inferred by PWGC are not simply spurious.

  Out-of-Sample estimates are performed on data that were not used in
  fitting the models.}
\end{figure}

We provide the final results in Figure
\ref{fig:logistic_regression_results} where the classification
accuracy is quantified by the (multiclass) MCC of $\approx 0.20$ on a held
out validation set consisting of $20\%$ of the data.  This MCC score
exceeds by a modest margin the performance of randomly guessing,
lending strong evidence to the assertion that PWGC successfully
recovers some meaningful differences between the Granger-causality
graphs of different subjects, particularly if we recall that there are
generally fewer examples for each subject than there are subjects
overall.  Moreover, distinguishing only between two particular
subjects (as opposed to distinguishing between one subject and $118$
others) is substantially easier -- an illustration is provided in
Figure \ref{fig:logistic_regression_2pair} where we embed the data
into the plane through supervised dimensionality reduction.  In
general, fitting a simple classifier between two subjects achieves an
$MCC$ in excess of $0.6$ and up to $0.9$, even when the same
hyperparameters are carried over for different pairs of subjects.

\begin{figure}
  \centering
  \caption{Subject Classification from Granger-causality Graphs}
  \begin{subfigure}[b]{0.45\textwidth}
    \caption{Distinguishing Individual Subjects}
    \label{fig:logistic_regression_2pair}
    \includegraphics[width=\linewidth]{logistic_regression_separation.pdf}
  \end{subfigure}
  \begin{subfigure}[b]{0.45\textwidth}
    \caption{Complete Multiclass Classifier}
    \label{fig:logistic_regression_results}
    \includegraphics[width=\linewidth]{logistic_regression_confusion_matrix.pdf}

    {\scriptsize }
  \end{subfigure}

  {\scriptsize Figure \ref{fig:logistic_regression_2pair} illustrates
    discriminating between two particular subjects based on their
    EEG Granger-causality graphs.  In this case $MCC = 0.75$ on held
    out data.  Visualization is constructed by supervised
    dimensionality reduction and is purely illustrative.

    Figure \ref{fig:logistic_regression_results} provides the
    row-normalized confusion matrix (computed on held out validation
    data) of a multiclass logistic regression classifier used to
    classify subjects based on their EEG Granger-causality graphs.  $MCC = 0.20$}

\end{figure}

\section{Conclusion}
\label{sec:conclusion}
In this paper we have applied the theoretical insights developed in
\cite{my_GC_paper} into a heuristic for estimating Granger-causality
graphs.  We have leveraged the notion of a strongly-causal graph which
enables the recover of causality graphs through simple pairwise
causality testing (Theorem \ref{thm:scg_recovery}), and demonstrated
that substantial improvements (both statistically and computationally)
over LASSO-based heuristics can be obtained.

We emphasize that the improvements are a result of topological
assumptions placed on the causality graph of the system, though
examples (see Figure \ref{fig:gene_networks}) from the literature
suggest that many graphs in practice may have special topological
properties that can be exploited as prior knowledge.  Further research
on the implications of Network properties (i.e. degree distributions,
or small-world properties \todo{cite some network theory}) for
Granger-causality is of interest.  Algorithms exploiting graphical
assumptions may, for example, take the form of specially developed
heuristics (e.g. Algorithm \ref{finite_pwgc}), cunning adaptive
weighting schemes for the LASSO, or formal Bayesian priors.

Our application in Section \ref{sec:application} provides strong
evidence that PWGC (Algorithm \ref{finite_pwgc}) is capable of
uncovering meaningful features from networks of time series data by
constructing a classifier which accurately discriminates between
subjects in an EEG experiment based purely on the Granger-causality
graph topology inferred by the iterated application of PWGC.  This is
a result which may be of independent scientific interest for EEG.

There are a number of potential extensions to this work.  Firstly, it
is known that the information theoretic notion of transfer entropy
reduces to Granger-causality when the data is Gaussian
\cite{barnett2009granger}, can Theorem \ref{thm:scg_recovery} be
generalized in a useful way to causality networks defined by transfer
entropy?  As well, the work of \cite{barnett2015granger} has
established the superiority of Granger-causality testing in state
space models (as opposed to pure autoregressions) in many cases.
Since, our pairwise recovery results do not require $x(t)$ to be
generated by a $\VAR$ model, such results can replace the Hypothesis
tests of Section \ref{sec:pairwise_hypothesis_testing} to enable
application to systems which are not well modelled by finite $\VAR$
models.

\clearpage
\printbibliography
\clearpage
\appendix

\section{Pairwise Causality Graph Estimation}
This section details efficient pairwise causality testing methodology
used in constructing the input data $F, P, \delta$ for Algorithm
\ref{alg:finite_pwgc}.  We employ the simplest reasonable methods in
order to maintain a focus on the Graph topological aspects.

\subsection{Pairwise Hypothesis Testing}
\label{sec:pairwise_hypothesis_testing}
In performing pairwise checks for Granger-causality $x_j \pwgc x_i$ we
follow the simple scheme of estimating the following two linear models:

\begin{align}
  H_0:&\ \widehat{x}_i^{(p)}(t) = \sum_{\tau = 1}^{p} b_{ii}(\tau)x_i(t - \tau),\\
  H_1:&\ \widehat{x}_i^{(p)}(t) = \sum_{\tau = 1}^{p} b_{ii}(\tau)x_i(t - \tau) + \sum_{\tau = 1}^pb_{ij}(\tau)x_j(t - \tau).
\end{align}

We formulate the statistic 

\begin{equation}
  \label{eqn:gc_statistics}
  F_{ij}(p) = \frac{T}{p}\Big(\frac{\xi_i(p)}{\xi_{ij}(p)} - 1\Big),
\end{equation}

where $\xi_i(p)$ is the sample mean square of the
residuals\footnote{This quantity is often denoted $\widehat{\sigma}$,
  but we maintain notation from Definition
  \ref{def:granger_causality}.}  $x_i(t) - \widehat{x}^{(p)}_i(t)$,

\begin{equation*}
  \xi_i(p) = \frac{1}{T - p}\sum_{t = p + 1}^T (x_i(t) - \widehat{x}_i^{(p)}(t))^2,
\end{equation*}

and similarly for $\xi_{ij}(p)$.  We test $F_{ij}(p)$ against a
$\chi^2(p)$ distribution.

If the estimation procedure is consistent, we will have the following
convergence (in $\P$ or a.s.):

\begin{equation}
  F_{ij}(p) \rightarrow
  \left\{
    \begin{array}{ll}
      0;\ x_j \npwgc x_i\\
      \infty;\ x_j \pwgc x_i
    \end{array}
  \right. \text{ as } T \rightarrow \infty.  % the '.' after \right is necessary.
\end{equation}

In our finite sample implementation (see Algorithm
\ref{alg:finite_pwgc}) we add edges to $\widehat{\gcg}$ in order of
the decreasing magnitude of $F_{ij}$ instead of proceeding backwards
through $P_{k - r}$ in Algorithm \ref{alg:pwgr}.  This makes greater
use of the information provided by the test statistic $F_{ij}$,
moreover, if $x_i \gc x_j$ and $x_j \gc x_k$, it is expected that
$F_{kj} > F_{ji}$, thereby providing the same effect as proceeding
backwards through $P_{k - r}$.
\subsection{Model Order Selection}
\label{sec:model_order_selection}
There are a variety of methods to choose the filter order $p$ (see
e.g. \cite{lutkepohl2005new}), but we will focus in particular on the
Bayesian Information Criteria (BIC).  The BIC is substantially more
conservative than the popular alternative Akaiake Information Criteria
(the BIC is also asymptotically consistent), and since we are
searching for \textit{sparse graphs}, we therefore prefer the BIC,
where we seek to \textit{minimize} over $p$:

\begin{equation}
  \label{eqn:bic}
  \begin{aligned}
    BIC_{\text{univariate}}(p) &= \ln\ \xi_i(p) + p\frac{\ln T}{T},\\
    BIC_{\text{bivariate}}(p) &= \ln \det \widehat{\Sigma}_{ij}(p) + 4p\frac{\ln T}{T},\\
  \end{aligned}
\end{equation}

where $\widehat{\Sigma}_{ij}(p)$ is the $2 \times 2$ residual
covariance matrix for the $\VAR(p)$ model of $(x_i(t), x_j(t))$.  The
bivariate errors $\xi_{ij}(p)$ and $\xi_{ji}(p)$ are the diagonal
entries of $\widehat{\Sigma}_{ij}(p)$.

We carry this out by a simple direct search on each model order
between $0$ and some prescribed $p_\text{max}$, resulting in a
collection $p_{ij}$ of model order estimates.  In practice, it is
sufficient to pick $p_\text{max}$ ad-hoc or via some simple heuristic
e.g. plotting the sequence $BIC(p)$ over $p$, though it is not
technically possible to guarantee that the optimal $p$ is less than
the chosen $p_\text{max}$ (since there can in general be arbitrarily
long lags from one variable to another).

\subsection{Efficient Model Estimation}
\label{sec:efficient_model_estimation}
In practice, the vast majority of computational effort involved in
implementing our estimation algorithm is spent calculating the error
estimates $\xi_i(p_i)$ and $\xi_{ij}(p_{ij})$.  This requires fitting a
total of $n^2p_{\text{max}}$ autoregressive models, where the most
naive algorithm (e.g. solving a least squares problem for each model)
for this task will consume $O(n^2p_{\text{max}}^4T)$ time, it is
possible to carry out this task in a much more modest
$O(n^2p_{\text{max}}^2 ) + O(n^2p_{\text{max}}T)$ time via the
autocorrelation method
\cite{hayes_statistical_digital_signal_processing} which substitutes
the following autocovariance estimates in the Yule-Walker
equations:\footnote{The particular indexing and normalization given in
  equation \ref{eqn:covariance_estimate} is critical to ensure
  $\widehat{R}$ is positive semidefinite.  The estimate can be viewed
  as calculating the covariance sequence of a signal multiplied by a
  rectangular window.}

\begin{equation}
  \label{eqn:covariance_estimate}
  \widehat{R}_x(\tau) = \frac{1}{T}\sum_{t = \tau + 1}^T x(t) x(t - \tau)^\T;\ \tau = 0, \ldots, p_{\text{max}},
\end{equation}

It is imperative that the first index in the summation is $\tau + 1$, as
opposed perhaps to $p_\text{max}$ and that the normalization is
$1 / T$, as opposed perhaps to $1 / (T - p_\text{max})$, in order to
guarantee that $\widehat{R}_x(\tau)$ forms a valid (i.e. positive
definite) covariance sequence.  This results in some bias, however the
dramatic computational speedup is worth it for our purposes.

These covariance estimates constitute the $O(n^2p_{\text{max}}T)$
operation.  Given these particular estimates, the variances $\xi_i(p)$
for $p = 1, \ldots, p_{\text{max}}$ can be evaluated in
$O(p_{\text{max}}^2)$ time each by applying the Levinson-Durbin
recursion to $\widehat{R}_{ii}(\tau)$, which effectively estimates a
sequence of $AR$ models, producing $\xi_i(p)$ as a side-effect (see
\cite{hayes_statistical_digital_signal_processing} and
\cite{levinson_durbin_recursion}).

Similarly, the variance estimates $\widehat{\Sigma}_{ij}(p)$ (which
include $\xi_{ij}$ and $\xi_{ji}$) can be obtained by estimating
$\frac{(n + 1)n}{2}$ bivariate AR models, again in
$O(p_{\text{max}}^2)$ time via Whittle's generalized Levinson-Durbin
recursion\footnote{We have made use of standalone tailor made
  implementations of these algorithms, available at
  \textsf{github.com/RJTK/Levinson-Durbin-Recursion}.}
\cite{whittle_generalized_levinson_durbin}.

\subsection{Edge Probabilities and Error Rate Controls}
\label{sec:error_rate_control}
Denote $F_{ij}$ the Granger-causality statistic of equation
\ref{eqn:gc_statistics} with model orders chosen by the methods of
Section \ref{sec:model_order_selection}.  We assume that this
statistic is asymptotically $\chi^2(p_{ij})$ distributed (the
disturbances are Gaussian), and denote by $G$ the cumulative
distribution function thereof.  We will define the matrix

\begin{equation}
  \label{eqn:edge_inclusion_probability}
  P_{ij} = G(F_{ij}),
\end{equation}

to be the matrix of pairwise edge inclusion P-values.  This is
motivated by the hypothesis test where the hypothesis $H_0$ will be
rejected (and thence we will conclude that $x_j \pwgc x_i$) if
$P_{ij} > 1 - \delta$.

The value $\delta$ can be chosen by a variety of methods, in our case
we apply the Benjamini Hochberg criteria \cite{benjamini_hochberg}
\cite{all_of_statistics} to control the false discovery rate of
pairwise edges to a level $\alpha$ (where we generally take
$\alpha = 0.05$).

\subsection{Finite Sample Recovery Algorithm}
\label{sec:finite_pwgc}

After the graph topology $\widehat{\gcg}$ has been estimated via
Algorithm \ref{alg:finite_pwgc}, we refit the entire model with the
specified sparsity pattern directly via ordinary least squares.

We note that producing graph estimates which are not strongly causal
can potentially be achieved by performing sequential estimates
$\widehat{x}_1(t), \widehat{x}_2(t), \ldots$ estimating a strongly causal
graph with the residuals of the previous model as input, and then
refitting on the combined sparsity pattern.  We experiment with this
heuristic in our example application of Section \ref{sec:application},
but reserve theoretical analysis for future work.

\clearpage
\printbibliography  
\end{document}

%%% Local Variables:
%%% mode: latex
%%% TeX-master: t
%%% End:
