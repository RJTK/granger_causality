\documentclass{beamer} %With pauses
% % \usepackage{fullpage}
\usepackage{framed}

% Figures
\usepackage{graphicx}
\usepackage{caption}
% \usepackage{subcaption}
\usepackage{wrapfig}
\usepackage{svg}

% Math packages, theorem definitions and numbering
\usepackage{amsmath}
\usepackage{amssymb}
\usepackage{amsthm}
\usepackage{mathrsfs} % Fancy scripted font
\usepackage{bm}  % Bold math
\usepackage{centernot}  % \centernot\implies looks better

% Misc packages
\usepackage[linesnumbered, ruled, vlined]{algorithm2e}
% \usepackage{algorithm2e} %{algorithm} environment
\usepackage{soul}  % \hl highlighting
\usepackage{color}
\usepackage{mathtools}  % For my \ceil function

% Theorems (with italics)
\theoremstyle{plain}  % Style definition removes italics
\newtheorem{theorem}{Theorem}
\newtheorem{corollary}{Corollary}
\newtheorem{proposition}{Proposition}
\newtheorem{lemma}{Lemma}

\theoremstyle{definition}
\newtheorem{remark}{Remark}
\newtheorem{definition}{Definition}
\newtheorem{example}{Example}
\newtheorem{assumption}{Assumption}

% keywords
\providecommand{\keywords}[1]{\textbf{\textit{Keywords---}} #1}

% General
\def\defeq{\overset{\Delta}{=}}  % Equal with triangle
\def\cl{\mathsf{cl\ }}  % Closure
\newcommand{\sgn}[1]{\mathsf{sgn}(#1)}  % sign function

% Calculus
\def\d{\mathsf{d}}  % Differential operator

% Functions
\def\ln{\mathsf{ln\ }}  % Natural logarithm
\DeclarePairedDelimiter{\ceil}{\lceil}{\rceil}  % Ceiling

% Probability
\def\H{\mathcal{H}}  % Hilbert space
\def\E{\mathbb{E}}  % Expectation
\def\Var{\text{Var}}  % Variance
\def\P{\mathbb{P}}  % Probability Measure
\def\F{\mathcal{F}}  % A sigma algebra
\def\sX{\mathcal{X}}  % Another sigma algebra
\def\KL{\mathbf{D}_{KL}}  % KL divergence
\def\bF{\mathbf{F}}  % Whole F-meas space
\def\GP{\mathcal{GP}}  % Gaussian process

% Standard sets
\def\Z{\mathbb{Z}}  % Set of integers
\def\R{\mathbb{R}}  % Set of real numbers
\def\C{\mathbb{C}}  % Set of complex numbers
\def\N{\mathbb{N}}  % Set of natural numbers
\def\ball{\mathbb{B}}  % Open ball
\def\clball{\overline{\ball}}  % Closed ball

% Linear algebra
\def\rk{\mathsf{rk }}  % The rank
\def\tr{\mathsf{tr }}  % The trace
\def\T{\mathsf{T}}  % Transpose notation
\def\c{\mathsf{c}}  % complement
\def\dg{\mathsf{dg }}   %  Diagonal vector of a matrix
\def\Dg{\mathsf{Dg }}   %  Diagonal matrix from a vector
\def\ind{\mathbf{1}}  % Ones vector or indicator
\def\matvec{\textbf{vec}}  % Vector operator
\def\<{\langle}  % < Inner product
\def\>{\rangle}  % > Inner product
\newcommand{\inner}[2]{\langle #1, #2 \rangle}  % Inner product
\newcommand{\innerT}[2]{#1^\T #2}  % Inner product for finite vectors

% Convex analysis
\def\conv{\mathsf{conv }}  % Convex hull
\def\prox{\mathsf{prox }}  % Proximity operator

% -----------------
% The given symbol or text (\text{mytext}) in a circle
% To be used always in math mode
\newcommand{\circlesign}[1]{ 
    \mathbin{
        \mathchoice
        {\buildcirclesign{\displaystyle}{#1}}
        {\buildcirclesign{\textstyle}{#1}}
        {\buildcirclesign{\scriptstyle}{#1}}
        {\buildcirclesign{\scriptscriptstyle}{#1}}
    } 
}

\newcommand\buildcirclesign[2]{%
    \begin{tikzpicture}[baseline=(X.base), inner sep=0, outer sep=0]
    \node[draw,circle] (X)  {\ensuremath{#1 #2}};
    \end{tikzpicture}%
}
% -----------------

%\documentclass[handout]{beamer} %Without pauses
\usepackage[backend=bibtex]{biblatex}
\usepackage{graphicx}
\usepackage{caption}
\usepackage{cancel}
\usefonttheme[onlymath]{serif}

\setbeamertemplate{bibliography item}{}
\setbeamertemplate{bibliography entry title}{}
\setbeamertemplate{bibliography entry location}{}
\setbeamertemplate{bibliography entry note}{}

\addtobeamertemplate{navigation symbols}{}{%
    \usebeamerfont{footline}%
    \usebeamercolor[fg]{footline}%
    \hspace{1em}%
    \insertframenumber/\inserttotalframenumber
}

\bibliography{\string~/Documents/academics/global_academics/global_bib.bib}

\usepackage{wasysym}
\usepackage{tikz}

\usetheme{PaloAlto}

\newtheorem*{defn}{Definition}

\def\E{\mathbb{E}}  % Expectation
\def\gc{\overset{\text{GC}}{\rightarrow}}  % Granger causality arrow
\def\ngc{\overset{\text{GC}}{\nrightarrow}}  % Negated Granger causality arrow
\def\pwgc{\overset{\text{PW}}{\rightarrow}}  % Pairwise Granger causality arrow
\def\npwgc{\overset{\text{PW}}{\nrightarrow}}  % Negated pairwise Granger causality arrow
\def\te{\overset{\mathcal{T}}{\rightarrow}}  % Transfer entropy arrow
\def\gcg{\mathcal{G}}  % Granger-causality graph
\def\gcge{\mathcal{E}}  % Graph edges
\def\VAR{\mathsf{VAR}}  % VAR(p) model
\def\B{\mathsf{B}}  % Filter B
\def\wtB{\widetilde{\B}}  % General filter B
\def\A{\mathsf{A}}  % Filter A
\def\H{\mathcal{H}}  % Hilbert space
\def\R{\mathbb{R}}  % Reals
\def\N{\mathbb{N}}  % Naturals
\def\T{\mathsf{T}}  % Transpose
\def\cl{\mathsf{cl}}  % Closure

\newcommand{\linE}[2]{\hat{\E}[#1\ |\ #2]}  % Linear projection
\newcommand{\linEerr}[2]{\xi[#1\ |\ #2]}  % Error of linear projection
\newcommand{\pa}[1]{pa(#1)}  % Parents of a node
\newcommand{\anc}[1]{\mathcal{A}(#1)}  % Ancestors of a node
\newcommand{\ancn}[2]{\mathcal{A}_{#1}(#2)}  % nth ancestors of a node
\newcommand{\gpn}[2]{gp_{#1}(#2)}  % nth generation grandparents
\newcommand{\wtalpha}[2]{\widetilde{\alpha}(#1, #2)}  % Some notation for lem:pwgc_anc
\newcommand{\dist}[2]{\mathsf{d}(#1, #2)}  % Distance between things
\newcommand{\gcgpath}[2]{#1 \rightarrow \cdots \rightarrow #2}  % A shorter path command

\title{Graph Topological Aspects of Granger Causal Network Learning}
\author{\texorpdfstring{Ryan J. Kinnear\newline\url{Ryan@Kinnear.ca}}{Ryan J. Kinnear}}% \author{Ryan J. Kinnear} \\
  % \small\href{mailto:ryan@kinnear.ca}{ryan@kinnear.ca} \\
  % \small\url{https://github.com/RJTK}\\
  % Under the Supervision of Ravi R. Mazumdar}

\institute[University of Waterloo] % (optional, but mostly needed)
{
  University of Waterloo\\
  Department of Electrical and Computer Engineering
}

\date{McGill University, Oct 4 2019}

\AtBeginSection[]
{
  \begin{frame}<beamer>{Outline}
    \tableofcontents[currentsection,currentsubsection]
  \end{frame}
}

% Let's get started
\begin{document}

\begin{frame}
  \titlepage
\end{frame}

\begin{frame}{Outline}
  \tableofcontents
  % You might wish to add the option [pausesections]
\end{frame}

\section{Introduction}

\begin{frame}{Causality}
  \begin{itemize}
    \item{Do billiard collisions cause motion?}\pause
    \item{Does standing in the street cause cars to stop?}\pause
    \item{Does smoking cause cancer?}\pause
    \item{Does government spending cause an increase in GDP?}\pause
    \item{Granger causality addresses the last question.}
  \end{itemize}
\end{frame}

\begin{frame}{Causality in Time Series}
  \begin{itemize}
    \item{Consider $x_t, y_t \overset{\text{i.i.d.}}{\sim} \text{BER}(1/2)$}\pause
    \item{If $z_t = x_{t - 1}$ then since cause preceeds effect, intuitively $x$ causes $z$}\pause
    \item{$z'_t = x_{t - 1} \oplus y_{t - 1}$ is more ambiguous.}\pause
    \item{Issue raised by: \fullcite{transfer_entropy_criticism}} 
  \end{itemize}
\end{frame}

\begin{frame}{Causality in Linear Time Series}
  \begin{itemize}
    \item{Granger causality formalizes ``cause proceeds effect'' for WSS processes}\pause
    \item{Interpretation as a ``Flow of Information'' or ``Flow of Energy'' is also reasonable}\pause
    \item{Given WSS processes, $x_1(t), \ldots, x_n(t)$, how can we understand causation $x_i \rightarrow x_j$?}\pause
    \item{\fullcite{granger1969investigating}}
  \end{itemize}
\end{frame}

\begin{frame}{Basic Setup}
  \begin{itemize}
    \item{Basic setup is an $n-$dimensional WSS process $x(t)$}\pause
    \item{Wold representation $x(t) = \sum_{\tau = 0}^\infty A(\tau)v(t - \tau)$}\pause
    \item{$\text{VAR}(\infty)$ representation $x(t) = \sum_{\tau = 1}^\infty B(\tau)x(t - \tau) + v(t)$}\pause
    \item{Require $\E[v(t)v(t - \tau)] = \delta_{\tau}\Sigma_v$, with $\Sigma_v$ Diagonal}\pause
  \end{itemize}

  \begin{defn}
    Linear Projection onto Hilbert space $\H \subseteq L_2(\Omega)$
    \begin{equation*}
      \linE{x}{\H} = \underset{u \in \H}{\text{argmin }} ||x - u||
    \end{equation*}

    Minimum Mean Squared Error
    \begin{equation*}
      \xi[x \ |\ \H] = \E (x - \linE{x}{\H})^2      
    \end{equation*}
  \end{defn}
\end{frame}

\begin{frame}{Basic Setup}
  More notation

  \begin{align*}
    \H_t &= \cl \{\sum_{\tau = 0}^p a_\tau^\T x(t - \tau)\ |\ a_\tau \in \R^n, p \in \N\} \subseteq L_2(\Omega)\\
    % H_t^x &= \{a x(t)\ |\ a \in \R\} \subseteq L_2(\Omega),
  \end{align*}\pause

  \begin{itemize}
    \item{$\H_t^{i}$ Hilbert space spanned by $x_i(t)$ alone}
    \item{$\H_t^{-i}$ Hilbert space spanned by $x(t)$ but \textit{excluding} $x_i(t)$}
  \end{itemize}
\end{frame}

\begin{frame}{Granger Causality}
  \begin{defn}
    For the WSS series $x(t)$ we say that component $x_j$
    \textit{Granger-Causes} component $x_i$ (with respect to $x$)
    and write $x_j \gc x_i$ if given Hilbert spaces $\H_{t - 1}$,
    $\H^{-j}_{t - 1}$ if

    \begin{equation*}
      \linEerr{x_i(t)}{\H_{t - 1}} < \linEerr{x_i(t)}{\H^{-j}_{t - 1}},
    \end{equation*}
  \end{defn}\pause

  \begin{defn}
    We say that $x_j$ pairwise Granger-causes $x_i$ and write
    $x_j \pwgc x_i$ if $x_j$ Granger-causes $x_i$ with respect only to
    $(x_i, x_j)$.
  \end{defn}
\end{frame}

\begin{frame}{Simple Examples}
  \begin{equation*}
    x_i(t) = b_i x(t - 1) + v_i(t); i = 0, 1, 2
  \end{equation*}

  \begin{figure}
    \includegraphics[width=0.75\linewidth]{./figures/var1_example.pdf}
  \end{figure}
\end{frame}

\begin{frame}{Simple Examples}
  \begin{align*}
    x_0(t) &= b_0 x_0(t - 1) + b_{0, 2} x_2(t - 150) + v_0(t)\\
    x_1(t) &= b_1 x_1(t - 1) + b_{1, 2} x_2(t - 150) + v_1(t)\\
    x_2(t) &= b_2 x_2(t - 1) + v_2(t)\\
  \end{align*}

  % TODO: Fix the whitespace consumed by this figure!

  \centering
  \includegraphics[width=0.75\linewidth]{./figures/var2_example.pdf}
\end{frame}

\section{Causality Graphs}
\begin{frame}{Graph Terminology}
  \begin{itemize}
    \item{Graph $\gcg = (V, \gcge)$ with nodes (or vertices) $V$ and edges $\gcge$}\pause
    \item{Directed paths: $j \rightarrow a_0 \rightarrow \cdots \rightarrow a_r \rightarrow i$ for $(a_{k}, a_{k + 1}) \in \gcge$ distinct nodes}\pause
    \item{Cycles: paths with $i = j$}
    \item{Directed Acyclic Graph (DAG): A directed graph without any cycles}\pause
    \item{Parents: $j \in \pa{i}$ iff $(j, i) \in \gcge$}\pause
    \item{Ancestors: $j \in \anc{i}$ iff $j \rightarrow i$ path in $\gcge$}\pause
  \end{itemize}
\end{frame}

\begin{frame}{Granger Causality Graph
  A Granger causality graph is simply a graph $\gcg = ([n], \gcge)$ with $(j, i) \in \gcge$ iff $j \gc i$.

  % TODO: Include a gcg example here, maybe a few papers
  % \includegraphics[]{}
    % \item{Confounder: A node $k$ confounds $i, j$ if $k \in \anc{i} \cap \anc{j}$ if exists a
    %     path $\gcgpath{k}{i}$ not containing $j$, and a path
    %     $\gcgpath{k}{j}$ not containing $i$.}
\end{frame}

\section{Pairwise Causal Flow}
\begin{frame}
\end{frame}

\section{Pairwise Testing Recovery Methods}
\begin{frame}
\end{frame}

\section{LASSO Recovery Methods}
\begin{frame}
\end{frame}

\end{document}
