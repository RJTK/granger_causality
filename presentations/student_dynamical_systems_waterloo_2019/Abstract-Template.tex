%    DO NOT CHANGE THE SETTINGS BELOW
%%%%%%%%%%%%%%%%%%%%%%%%%%%%%%%%%%%%%%%%
\documentclass[12pt]{article}
\usepackage{epsfig}
\usepackage{graphicx,float}
\usepackage{url}
\usepackage[hidelinks,colorlinks=true,bookmarks=true]{hyperref}
%*******************************************
\setlength{\textheight}{24cm}		
\setlength{\textwidth}{17.5cm}		
\setlength{\oddsidemargin}{-0.5cm}
\setlength{\topmargin}{-1.25cm}
%*******************************************
 \pagestyle{empty} 

\begin{document}
\hspace{-0.70cm}\rule{17.5cm}{0.01 in} \\
\vspace{-0.4cm}
\begin{flushleft}
\Large \textbf{\noindent
Insert the Title of Your Talk Here}
%%%%%%%%%%%%%%%%%%%%%%%%%%%%%%%%%%%%%%%%%%%%%%%%%%%%%%%%
\\
%%%%%%%%%%%%%%%%%%%%%%%%%%%%%%%%%%%%%%%%%%%%%%%%%%%%%%%%
\vspace{0.5cm}
\normalsize
 %%%%%%             NAMES OF AUTHORS         %%%%%%
 %%%%%% NAME OF SPEAKER SHOULD BE UNDERLINED %%%%%%%%%
\normalsize{
 \underline{Ryan J. Kinnear}$^1$, Ravi R. Mazumdar$^1$
} \\
\vspace{5mm}
\textit{\footnotesize
 %%%%%% AFFILIATION OF AUTHORS %%%%%%
$^1$ Department of Electrical and Computer Engineering, University of Waterloo, Waterloo, Ontario N2L 3G1, Canada\\
 %%%%%%%%%%%%%%%%%%%%%%%%%%%%%%%%%%%%%
}
\end{flushleft}

We study Granger causality in the context of wide-sense stationary
time series, where our focus is on the topological aspects of the
underlying causality graph.  We establish sufficient conditions (in
particular, we develop the notion of a ``strongly causal'' graph
topology) under which the true causality graph can be recovered via
pairwise causality testing alone, and provide examples from the gene
regulatory network literature suggesting that our concept of a
strongly causal graph may be applicable to this field.  We implement
and detail finite-sample heuristics derived from our theory, and
establish through simulation the efficiency gains (both statistical
and computational) which can be obtained (in comparison to LASSO-type
algorithms) when structural assumptions are met.

In this paper we study the notion of Granger causality
\cite{granger1969investigating} \cite{Granger1980329} as a means of
uncovering an underlying causal structure in multivariate time series.
Though the underlying causality graph cannot be observed directly,
it's presence is inferred as a latent structure among observed
time series data.  This notion is leveraged in a variety of
applications e.g. in Neuroscience as a means of recovering
interactions amongst brain regions \cite{bressler2011wiener},
\cite{anna_paper2008}, \cite{david2008identifying}; in the study of
the dependence and connectedness of financial institutions
\cite{NBERw16223}; gene expression networks \cite{Fujita2007},
\cite{methods_for_inferring_gene_regulatory_networks_from_time_series_expression_data},
\cite{grouped_graphical_granger_modelling_for_gene_expression_regulatory_networks_discovery},
\cite{discovering_graphical_Granger_causality_using_the_truncating_lasso_penalty};
and power system design \cite{Misyrlis2016450}, \cite{yuan2014root}.

The principal contributions of this paper are as follows: firstly, in
section \ref{sec:theory} we study \textit{pairwise} Granger causality
relations, providing novel theorems connecting the structure of the
causality graph to the pairwise ``causality flow'' in the system, as
well as an interpretation in terms of the graph topology of the
sparsity pattern of matrices arising in the Wold decomposition,
generalizing in some sense the notion of ``feedback-free'' processes
studied by \cite{caines1975feedback} in close connection with
Granger causality.  We establish sufficient conditions (sections
\ref{sec:strongly_causal_graphs}, \ref{sec:persistent_systems}) under
which a fully conditional Granger causality graph can be recovered
from pairwise tests alone (sec \ref{sec:pairwise_algorithm}).  We
report a summary of simulation results in \ref{sec:simulation}, with
additional results reported in the supplementary material Section
\ref{apx:simulation}.  Our simulation results establish that there is
significant potential for improvement over existing methods, and that
the graph-topological aspects of time series analysis are relevant for
both theory and practice.  Concluding remarks on further open problems
and extensions are provided in Section \ref{sec:conclusion}.  The
proofs of each proposition and theorem are also relegated to the
supplementary material, simple corollaries have proofs included in the
main text.

In a general Granger causality graph, there are natural
\textit{necessary} conditions for a node $j$ to pairwise cause node
$i$, namely that $j$ is an ancestor to $i$ or that there is a third
confounding node $k$.  On the other hand, we will see simple examples
for why \textit{sufficient} conditions are more nuanced: if $j$ causes
$i$ conditionally on all other nodes in the graph, it need not be the
case that $j$ causes $i$ when the two are observed in isolation.  We
examine a special \textit{graph topological} assumption we term
``strongly causal'' wherein a natural converse \textit{does} hold,
satisfying an intuitive notion of ``causality flow'' through the
graph.  We highlight examples from the literature on gene networks
suggesting that assumptions of strongly causal graph structure may not
be out of place in some application areas.

As a consequence of the causality graph being strongly causal, we show
that it is possible to recover the full conditional causality graph
through pairwise testing alone -- that is, it isn't necessary to
perform any conditional causality tests.

% $\bullet$ Abstracts may contain equations: Equations are to be centered horizontally, with equation numbers aligned with the right margin. See equation (\ref{eq1})
% \begin{equation}
% e^{i\pi}+1=0.\label{eq1}
% \end{equation}




% $\bullet$ Abstracts must contain at least two  references. See   \cite{ASY,STE}.


% $\bullet$ Abstracts may contain figures. For example, Figure \ref{fig1}.

\begin{figure}[H]
\centering
\framebox[6cm]{\rule{0cm}{0.5cm} Abstracts may contain figures}
%\includegraphics[width =0.48\linewidth]{image} 
\caption{Figure one caption.}
\label{fig1}
\end{figure}
 
 
 
 Please send your source file  and .pdf file to Alyssa Novelia (anovelia@uwaterloo.ca) or Isam Al-Darabsah (ialdarabsah@uwaterloo.ca)


 

%%%%%%%%%%%%%%%%%%%%%%%%%%%%%%%%%%%%%%%%
%% BIBLIOGRAPHY
%%%%%%%%%%%%%%%%%%%%%%%%%%%%%%%%%%%%%%%%
\begin{thebibliography}{00}
\addcontentsline{toc}{chapter}{References}
\setlength{\itemsep}{-1mm}
\small
\bibitem{ASY} K.T. Alligood, T.D. Sauer and J. A. Yorke. \textit{Chaos: An Introduction to Dynamical Systems}, New York, Springer-Verlag (1997).

\bibitem{STE} L. Schimansky-Geier, A.V. Tolstopjatenko and W. Ebelin \textit{Noise induced transitions due to external additive noise}, Phys. Lett. A, \textbf{108}, 7, pp. 329-332 (1985).
\end{thebibliography}
\normalsize
\end{document} 
