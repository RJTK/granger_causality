\documentclass[12pt]{article}

%% bibliography stuff -- this needs come before the preamble inclusion
\usepackage[backend=bibtex,sorting=none]{biblatex}
\bibliography{\string~/Documents/academics/global_academics/global_bib.bib}
\usepackage{hyperref}

% \usepackage{fullpage}
\usepackage{framed}

% Figures
\usepackage{graphicx}
\usepackage{caption}
% \usepackage{subcaption}
\usepackage{wrapfig}
\usepackage{svg}

% Math packages, theorem definitions and numbering
\usepackage{amsmath}
\usepackage{amssymb}
\usepackage{amsthm}
\usepackage{mathrsfs} % Fancy scripted font
\usepackage{bm}  % Bold math
\usepackage{centernot}  % \centernot\implies looks better

% Misc packages
\usepackage[linesnumbered, ruled, vlined]{algorithm2e}
% \usepackage{algorithm2e} %{algorithm} environment
\usepackage{soul}  % \hl highlighting
\usepackage{color}
\usepackage{mathtools}  % For my \ceil function

% Theorems (with italics)
\theoremstyle{plain}  % Style definition removes italics
\newtheorem{theorem}{Theorem}
\newtheorem{corollary}{Corollary}
\newtheorem{proposition}{Proposition}
\newtheorem{lemma}{Lemma}

\theoremstyle{definition}
\newtheorem{remark}{Remark}
\newtheorem{definition}{Definition}
\newtheorem{example}{Example}
\newtheorem{assumption}{Assumption}

% keywords
\providecommand{\keywords}[1]{\textbf{\textit{Keywords---}} #1}

% General
\def\defeq{\overset{\Delta}{=}}  % Equal with triangle
\def\cl{\mathsf{cl\ }}  % Closure
\newcommand{\sgn}[1]{\mathsf{sgn}(#1)}  % sign function

% Calculus
\def\d{\mathsf{d}}  % Differential operator

% Functions
\def\ln{\mathsf{ln\ }}  % Natural logarithm
\DeclarePairedDelimiter{\ceil}{\lceil}{\rceil}  % Ceiling

% Probability
\def\H{\mathcal{H}}  % Hilbert space
\def\E{\mathbb{E}}  % Expectation
\def\Var{\text{Var}}  % Variance
\def\P{\mathbb{P}}  % Probability Measure
\def\F{\mathcal{F}}  % A sigma algebra
\def\sX{\mathcal{X}}  % Another sigma algebra
\def\KL{\mathbf{D}_{KL}}  % KL divergence
\def\bF{\mathbf{F}}  % Whole F-meas space
\def\GP{\mathcal{GP}}  % Gaussian process

% Standard sets
\def\Z{\mathbb{Z}}  % Set of integers
\def\R{\mathbb{R}}  % Set of real numbers
\def\C{\mathbb{C}}  % Set of complex numbers
\def\N{\mathbb{N}}  % Set of natural numbers
\def\ball{\mathbb{B}}  % Open ball
\def\clball{\overline{\ball}}  % Closed ball

% Linear algebra
\def\rk{\mathsf{rk }}  % The rank
\def\tr{\mathsf{tr }}  % The trace
\def\T{\mathsf{T}}  % Transpose notation
\def\c{\mathsf{c}}  % complement
\def\dg{\mathsf{dg }}   %  Diagonal vector of a matrix
\def\Dg{\mathsf{Dg }}   %  Diagonal matrix from a vector
\def\ind{\mathbf{1}}  % Ones vector or indicator
\def\matvec{\textbf{vec}}  % Vector operator
\def\<{\langle}  % < Inner product
\def\>{\rangle}  % > Inner product
\newcommand{\inner}[2]{\langle #1, #2 \rangle}  % Inner product
\newcommand{\innerT}[2]{#1^\T #2}  % Inner product for finite vectors

% Convex analysis
\def\conv{\mathsf{conv }}  % Convex hull
\def\prox{\mathsf{prox }}  % Proximity operator

% -----------------
% The given symbol or text (\text{mytext}) in a circle
% To be used always in math mode
\newcommand{\circlesign}[1]{ 
    \mathbin{
        \mathchoice
        {\buildcirclesign{\displaystyle}{#1}}
        {\buildcirclesign{\textstyle}{#1}}
        {\buildcirclesign{\scriptstyle}{#1}}
        {\buildcirclesign{\scriptscriptstyle}{#1}}
    } 
}

\newcommand\buildcirclesign[2]{%
    \begin{tikzpicture}[baseline=(X.base), inner sep=0, outer sep=0]
    \node[draw,circle] (X)  {\ensuremath{#1 #2}};
    \end{tikzpicture}%
}
% -----------------


\graphicspath{{./figures/}}

\title{Structure Learning for $VAR(p)$ Models}
\author{R. J. Kinnear, R. R. Mazumdar}

\begin{document}
\maketitle
\abstract{We study Granger Causality and propose a structure learning
  heuristic for uncovering a parsimonious representation of large
  $\mathsf{VAR}(p)$ models.}

\section{Introduction and Review}
Consider a collection of stochastic processes producing observations
at discrete time intervals.  Are the underlying processes dependent?
Can we quantify any of the underlying relationships?  Can the arrow of
time help us to distinguish a directionality or flow of dependence
among our observed series?  In this paper we contribute to the
understanding of the notion of Granger-Causality
\cite{granger1969investigating} as a tool for answering these questions.

Though the notion of causality is a philosophically slippery concept,
it is fundamental to the way we understand the world and to the
progress of science in general.  Indeed, without faith in the
consistency of causal interactions the results of experimental science
could not be generalized or applied in any meaningful way.  In the
case of Granger-Causality, we state that if an event '$A$' provides us
with unique (that is, not available anywhere else) information about a
later event $B$, then $A$ must have a causal impact on $B$.  As
opposed to the notion of Causation promoted by Pearl
\cite{pearl2000art}, this is an entirely model-free notion of cause,
and instead leverages the intuition that a cause must precede it's
effect.

In practice, Granger's notion of causation it not a convincing test
for \textit{true} causation, since our statements about causation are
highly dependent upon the data that we are able to observe.  We prefer
instead to interpret Granger causality as a means of uncovering a flow
of ``information'' or ``energy'' through some underlying graph of
interactions.  Though this graph cannot be observed directly, we will
infer it's presence as a latent structure among our observed time
series data.

\section{Theory}
\section{Structure Learning} 
\section{Application}
\section{Conclusion}

\printbibliography
\end{document}

%%% Local Variables:
%%% mode: latex
%%% TeX-master: t
%%% End:
